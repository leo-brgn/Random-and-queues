% Options for packages loaded elsewhere
\PassOptionsToPackage{unicode}{hyperref}
\PassOptionsToPackage{hyphens}{url}
%
\documentclass[
]{article}
\usepackage{amsmath,amssymb}
\usepackage{lmodern}
\usepackage{iftex}
\ifPDFTeX
  \usepackage[T1]{fontenc}
  \usepackage[utf8]{inputenc}
  \usepackage{textcomp} % provide euro and other symbols
\else % if luatex or xetex
  \usepackage{unicode-math}
  \defaultfontfeatures{Scale=MatchLowercase}
  \defaultfontfeatures[\rmfamily]{Ligatures=TeX,Scale=1}
\fi
% Use upquote if available, for straight quotes in verbatim environments
\IfFileExists{upquote.sty}{\usepackage{upquote}}{}
\IfFileExists{microtype.sty}{% use microtype if available
  \usepackage[]{microtype}
  \UseMicrotypeSet[protrusion]{basicmath} % disable protrusion for tt fonts
}{}
\makeatletter
\@ifundefined{KOMAClassName}{% if non-KOMA class
  \IfFileExists{parskip.sty}{%
    \usepackage{parskip}
  }{% else
    \setlength{\parindent}{0pt}
    \setlength{\parskip}{6pt plus 2pt minus 1pt}}
}{% if KOMA class
  \KOMAoptions{parskip=half}}
\makeatother
\usepackage{xcolor}
\IfFileExists{xurl.sty}{\usepackage{xurl}}{} % add URL line breaks if available
\IfFileExists{bookmark.sty}{\usepackage{bookmark}}{\usepackage{hyperref}}
\hypersetup{
  pdftitle={Rapport},
  hidelinks,
  pdfcreator={LaTeX via pandoc}}
\urlstyle{same} % disable monospaced font for URLs
\usepackage[margin=1in]{geometry}
\usepackage{color}
\usepackage{fancyvrb}
\newcommand{\VerbBar}{|}
\newcommand{\VERB}{\Verb[commandchars=\\\{\}]}
\DefineVerbatimEnvironment{Highlighting}{Verbatim}{commandchars=\\\{\}}
% Add ',fontsize=\small' for more characters per line
\usepackage{framed}
\definecolor{shadecolor}{RGB}{248,248,248}
\newenvironment{Shaded}{\begin{snugshade}}{\end{snugshade}}
\newcommand{\AlertTok}[1]{\textcolor[rgb]{0.94,0.16,0.16}{#1}}
\newcommand{\AnnotationTok}[1]{\textcolor[rgb]{0.56,0.35,0.01}{\textbf{\textit{#1}}}}
\newcommand{\AttributeTok}[1]{\textcolor[rgb]{0.77,0.63,0.00}{#1}}
\newcommand{\BaseNTok}[1]{\textcolor[rgb]{0.00,0.00,0.81}{#1}}
\newcommand{\BuiltInTok}[1]{#1}
\newcommand{\CharTok}[1]{\textcolor[rgb]{0.31,0.60,0.02}{#1}}
\newcommand{\CommentTok}[1]{\textcolor[rgb]{0.56,0.35,0.01}{\textit{#1}}}
\newcommand{\CommentVarTok}[1]{\textcolor[rgb]{0.56,0.35,0.01}{\textbf{\textit{#1}}}}
\newcommand{\ConstantTok}[1]{\textcolor[rgb]{0.00,0.00,0.00}{#1}}
\newcommand{\ControlFlowTok}[1]{\textcolor[rgb]{0.13,0.29,0.53}{\textbf{#1}}}
\newcommand{\DataTypeTok}[1]{\textcolor[rgb]{0.13,0.29,0.53}{#1}}
\newcommand{\DecValTok}[1]{\textcolor[rgb]{0.00,0.00,0.81}{#1}}
\newcommand{\DocumentationTok}[1]{\textcolor[rgb]{0.56,0.35,0.01}{\textbf{\textit{#1}}}}
\newcommand{\ErrorTok}[1]{\textcolor[rgb]{0.64,0.00,0.00}{\textbf{#1}}}
\newcommand{\ExtensionTok}[1]{#1}
\newcommand{\FloatTok}[1]{\textcolor[rgb]{0.00,0.00,0.81}{#1}}
\newcommand{\FunctionTok}[1]{\textcolor[rgb]{0.00,0.00,0.00}{#1}}
\newcommand{\ImportTok}[1]{#1}
\newcommand{\InformationTok}[1]{\textcolor[rgb]{0.56,0.35,0.01}{\textbf{\textit{#1}}}}
\newcommand{\KeywordTok}[1]{\textcolor[rgb]{0.13,0.29,0.53}{\textbf{#1}}}
\newcommand{\NormalTok}[1]{#1}
\newcommand{\OperatorTok}[1]{\textcolor[rgb]{0.81,0.36,0.00}{\textbf{#1}}}
\newcommand{\OtherTok}[1]{\textcolor[rgb]{0.56,0.35,0.01}{#1}}
\newcommand{\PreprocessorTok}[1]{\textcolor[rgb]{0.56,0.35,0.01}{\textit{#1}}}
\newcommand{\RegionMarkerTok}[1]{#1}
\newcommand{\SpecialCharTok}[1]{\textcolor[rgb]{0.00,0.00,0.00}{#1}}
\newcommand{\SpecialStringTok}[1]{\textcolor[rgb]{0.31,0.60,0.02}{#1}}
\newcommand{\StringTok}[1]{\textcolor[rgb]{0.31,0.60,0.02}{#1}}
\newcommand{\VariableTok}[1]{\textcolor[rgb]{0.00,0.00,0.00}{#1}}
\newcommand{\VerbatimStringTok}[1]{\textcolor[rgb]{0.31,0.60,0.02}{#1}}
\newcommand{\WarningTok}[1]{\textcolor[rgb]{0.56,0.35,0.01}{\textbf{\textit{#1}}}}
\usepackage{longtable,booktabs,array}
\usepackage{calc} % for calculating minipage widths
% Correct order of tables after \paragraph or \subparagraph
\usepackage{etoolbox}
\makeatletter
\patchcmd\longtable{\par}{\if@noskipsec\mbox{}\fi\par}{}{}
\makeatother
% Allow footnotes in longtable head/foot
\IfFileExists{footnotehyper.sty}{\usepackage{footnotehyper}}{\usepackage{footnote}}
\makesavenoteenv{longtable}
\usepackage{graphicx}
\makeatletter
\def\maxwidth{\ifdim\Gin@nat@width>\linewidth\linewidth\else\Gin@nat@width\fi}
\def\maxheight{\ifdim\Gin@nat@height>\textheight\textheight\else\Gin@nat@height\fi}
\makeatother
% Scale images if necessary, so that they will not overflow the page
% margins by default, and it is still possible to overwrite the defaults
% using explicit options in \includegraphics[width, height, ...]{}
\setkeys{Gin}{width=\maxwidth,height=\maxheight,keepaspectratio}
% Set default figure placement to htbp
\makeatletter
\def\fps@figure{htbp}
\makeatother
\setlength{\emergencystretch}{3em} % prevent overfull lines
\providecommand{\tightlist}{%
  \setlength{\itemsep}{0pt}\setlength{\parskip}{0pt}}
\setcounter{secnumdepth}{-\maxdimen} % remove section numbering
\ifLuaTeX
  \usepackage{selnolig}  % disable illegal ligatures
\fi

\title{Rapport}
\author{}
\date{\vspace{-2.5em}2022-03-07}

\begin{document}
\maketitle

\hypertarget{partie-1}{%
\subsection{Partie 1}\label{partie-1}}

\hypertarget{question-2.1}{%
\subsubsection{Question 2.1}\label{question-2.1}}

\includegraphics{Rapport_files/figure-latex/unnamed-chunk-1-1.pdf}

La méthode de Von Neumann ne semble pas satisfaisante pour générer des
nombres aléatoires suivant une loi uniforme. On peut remarquer des
valeurs absorbantes. Les trois autres méthodes semblent quant à elles
fournir des résultats plutôt satisfaisants. On peut supposer d'après le
graphique que l'écart-type est assez faible pour ces trois méthodes.

\hypertarget{question-2.2}{%
\subsubsection{Question 2.2}\label{question-2.2}}

\includegraphics{Rapport_files/figure-latex/unnamed-chunk-2-1.pdf}

Ce mode de visualisation nous permet de voir si la valeur donnée par
l'algorithme dépend ou non de la valeur précédente. On remarque que pour
tous les algorithmes sauf celui de Von Neumann, il ne semble pas y avoir
de corrélation entre les valeurs de rang N et N+1. En effet, on ne voit
pas de courbe ou de droite se dessiner et les valeurs sont au contraire
très dispersées sur les graphiques.

\hypertarget{question-3}{%
\subsubsection{Question 3}\label{question-3}}

\begin{Shaded}
\begin{Highlighting}[]
\NormalTok{fr\_mt }\OtherTok{\textless{}{-}} \FunctionTok{Frequency}\NormalTok{(mt,}\DecValTok{32}\NormalTok{)}
\NormalTok{fr\_vn }\OtherTok{\textless{}{-}} \FunctionTok{Frequency}\NormalTok{(vn,}\DecValTok{14}\NormalTok{)}
\NormalTok{fr\_ru }\OtherTok{\textless{}{-}} \FunctionTok{Frequency}\NormalTok{(ru,}\DecValTok{31}\NormalTok{)}
\NormalTok{fr\_stm }\OtherTok{\textless{}{-}} \FunctionTok{Frequency}\NormalTok{(stm,}\DecValTok{31}\NormalTok{)}
\NormalTok{p\_mt }\OtherTok{\textless{}{-}} \DecValTok{2} \SpecialCharTok{*}\NormalTok{ (}\DecValTok{1}\SpecialCharTok{{-}}\FunctionTok{pnorm}\NormalTok{(fr\_mt))}
\NormalTok{p\_mt }\OtherTok{\textless{}{-}} \FunctionTok{sum}\NormalTok{(p\_mt)}\SpecialCharTok{/}\FunctionTok{length}\NormalTok{(p\_mt)}
\NormalTok{p\_vn }\OtherTok{\textless{}{-}} \DecValTok{2} \SpecialCharTok{*}\NormalTok{ (}\DecValTok{1}\SpecialCharTok{{-}}\FunctionTok{pnorm}\NormalTok{(fr\_vn))}
\NormalTok{p\_vn }\OtherTok{\textless{}{-}} \FunctionTok{sum}\NormalTok{(p\_vn)}\SpecialCharTok{/}\FunctionTok{length}\NormalTok{(p\_vn)}
\NormalTok{p\_ru }\OtherTok{\textless{}{-}} \DecValTok{2} \SpecialCharTok{*}\NormalTok{ (}\DecValTok{1}\SpecialCharTok{{-}}\FunctionTok{pnorm}\NormalTok{(fr\_ru))}
\NormalTok{p\_ru }\OtherTok{\textless{}{-}} \FunctionTok{sum}\NormalTok{(p\_ru)}\SpecialCharTok{/}\FunctionTok{length}\NormalTok{(p\_ru)}
\NormalTok{p\_stm }\OtherTok{\textless{}{-}} \DecValTok{2} \SpecialCharTok{*}\NormalTok{ (}\DecValTok{1}\SpecialCharTok{{-}}\FunctionTok{pnorm}\NormalTok{(fr\_stm))}
\NormalTok{p\_stm}\OtherTok{\textless{}{-}} \FunctionTok{sum}\NormalTok{(p\_stm)}\SpecialCharTok{/}\FunctionTok{length}\NormalTok{(p\_stm)}
\end{Highlighting}
\end{Shaded}

\begin{longtable}[]{@{}
  >{\raggedright\arraybackslash}p{(\columnwidth - 8\tabcolsep) * \real{0.1765}}
  >{\raggedright\arraybackslash}p{(\columnwidth - 8\tabcolsep) * \real{0.1912}}
  >{\raggedright\arraybackslash}p{(\columnwidth - 8\tabcolsep) * \real{0.2647}}
  >{\raggedright\arraybackslash}p{(\columnwidth - 8\tabcolsep) * \real{0.1029}}
  >{\raggedright\arraybackslash}p{(\columnwidth - 8\tabcolsep) * \real{0.2647}}@{}}
\toprule
\begin{minipage}[b]{\linewidth}\raggedright
Algorithme
\end{minipage} & \begin{minipage}[b]{\linewidth}\raggedright
Von Neumann
\end{minipage} & \begin{minipage}[b]{\linewidth}\raggedright
Mersenne Twister
\end{minipage} & \begin{minipage}[b]{\linewidth}\raggedright
RANDU
\end{minipage} & \begin{minipage}[b]{\linewidth}\raggedright
Standard minimal
\end{minipage} \\
\midrule
\endhead
P\_valeur & 0.00018 & 0.507 & 0.506 & 0.467 \\
\bottomrule
\end{longtable}

Pour valider l'hypothèse que l'algorithme produit une séquence
aléatoire, il faut que P\_valeur soit strictement supérieur à 0.01. De
ce fait, l'algorithme de Von Neumann n'est clairement pas satisfaisant.
Toutefois les trois autres algorithmes le sont.

\hypertarget{question-4}{%
\subsubsection{Question 4}\label{question-4}}

\end{document}
